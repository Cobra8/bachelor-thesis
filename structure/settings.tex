% !TEX root = ../main.tex

%----------------------------------------------------------------------------------------
% PACKAGE CONFIGURATIONS
%----------------------------------------------------------------------------------------

% Filename of the bibliography
\addbibresource{structure/main.bib}

% Margin settings
\geometry{
  paper=a4paper, % Paper format
  inner=2.5cm, % Inner margin
  outer=3.8cm, % Outer margin
  bindingoffset=.5cm, % Binding offset
  top=1.5cm, % Top margin
  bottom=1.5cm, % Bottom margin
  %showframe, % Uncomment to show how the type block is set on the page
}

% Figures location
\graphicspath{{figures/}}
\DeclareGraphicsExtensions{.pdf,.png,.jpg,.jpeg,.eps,.ps}

% caption setup
\captionsetup[figure]{position=bottom,width=.9\linewidth}

% Custom P4 syntax hightlighting
\lstdefinelanguage{P4}{
    morekeywords={ struct, package, header, metadata, parser, transition, state, actions, action, table, key, control, extern, verif, if, else if, else, return, hit, miss, true, false, inout, in, out, exit, and, or, exact, ternary, lpm, range, valid, mask, match_kind },
    emph={void, const, bool, varbit, bit, int, typedef, tuple, enum, apply },
    sensitive=false, % keywords are not case-sensitive
    morecomment=[l]{//}, % l is for line comment
    morecomment=[s]{/*}{*/}, % s is for start and end delimiter
    morestring=[b]" % defines that strings are enclosed in double quotes
}

% P4 syntax hightlighting (use with \begin{P4} or \begin{lstlisting}[style=P4])
\colorlet{p4gray}{black!30}
\colorlet{p4green}{green!60!blue}
\colorlet{p4key}{brown}
\colorlet{p4emph}{violet}
\colorlet{p4mauve}{red!60!blue}
\lstdefinestyle{P4}{
    backgroundcolor=\color{gray!10},
    basicstyle=\tiny\ttfamily,
    columns=fullflexible,
    breakatwhitespace=false,
    breaklines=true,
    captionpos=b,
    commentstyle=\color{p4green},
    extendedchars=true,
    frame=single,
    keepspaces=true,
    keywordstyle=\color{p4key},
    emphstyle=\color{p4emph},
    language=P4,
    numbers=none, % line numbers (left, right or none)
    numbersep=0,
    numberstyle=\tiny\color{blue},
    rulecolor=\color{p4gray},
    showspaces=false,
    showtabs=false,
    stepnumber=1,
    stringstyle=\color{p4mauve},
    tabsize=3,
    title=\lstname
}

\lstnewenvironment{P4}{
    \lstset{style=P4}
  }
  {}

% C++ syntax hightlighting (use with \begin{C++} or \begin{lstlisting}[style=C++])
\colorlet{cppgray}{black!30}
\colorlet{cppgreen}{green!60!blue}
\colorlet{cppmauve}{red!60!blue}
\lstdefinestyle{C++}{
    backgroundcolor=\color{gray!10},
    basicstyle=\tiny\ttfamily,
    columns=fullflexible,
    breakatwhitespace=false,
    breaklines=true,
    captionpos=b,
    commentstyle=\color{cppgreen},
    extendedchars=true,
    frame=single,
    keepspaces=true,
    keywordstyle=\color{blue},
    language=c++,
    numbers=none,
    numbersep=5pt,
    numberstyle=\tiny\color{blue},
    rulecolor=\color{cppgray},
    showspaces=false,
    showtabs=false,
    stepnumber=5,
    stringstyle=\color{cppmauve},
    tabsize=3,
    title=\lstname
}

\lstnewenvironment{C++}{
    \lstset{style=C++}
  }
  {}

% Python syntax hightlighting (use with \begin{python} or \begin{lstlisting}[style=python])
\definecolor{pythonblue}{rgb}{0,0,0.5}
\definecolor{pythonred}{rgb}{0.6,0,0}
\definecolor{pythongreen}{rgb}{0,0.5,0}
\colorlet{pythongray}{black!30}
\lstdefinestyle{python}{
    backgroundcolor=\color{gray!10},
    basicstyle=\tiny\ttfamily,
    columns=fullflexible,
    breakatwhitespace=false,
    breaklines=true,
    captionpos=b,
    commentstyle=\color{pythongray},
    extendedchars=true,
    frame=single,
    keepspaces=true,
    keywordstyle=\color{pythonblue},
    morekeywords={access,and,break,class,continue,def,del,elif,else,except,exec,finally,for,from,global,if,import,in,is,lambda,not,or,pass,print,raise,return,try,while},
    language=python,
    numbers=none,
    numbersep=5pt,
    numberstyle=\tiny\color{pythonred},
    rulecolor=\color{pythongray},
    showspaces=false,
    showtabs=false,
    stepnumber=5,
    stringstyle=\color{pythongreen},
    showstringspaces=false,
    tabsize=3,
    title=\lstname
}

\lstnewenvironment{python}{
    \lstset{
      style=python,
      emph={BATCH_SIZE,SIGN_MASK,EXPONENT_MASK,MANTISSA_MASK},
      emphstyle=\color{pythonred},
    }
  }
  {}
