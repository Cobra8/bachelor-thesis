% !TEX root = ../main.tex

\chapter{RMI for P4}
\label{ch:rmiforp4}

\section{RMI reference implementation}
The RMI reference implementation following from \cite{cdfshop} available at \cite{rmi-reference} is implemented in Rust and primarily serves as a compiler that takes in a dataset as input and outputs C++ source code files. One can play with multiple hyperparameters to influence the generated code and with that the potential performance of the generated implementation. Concretely, there is the possibility to choose which model type is used on which level as well as a parameter called the branching factor that determines the number of leaf models between two layers. The reference implementation currently supports nine different model types the most frequently used ones being linear and cubic. The functionality of the implementation does not stop at this point. Instead there is a possibility to pass an optimize option to the executable to let RMI perform automatic tuning that outputs a table that covers heuristically selected possible RMI configurations that cover the Pareto front. This table then contains different suggestions for which combination of models can be used together with a branching factor as well as information about the layer parameter size and approximately how many binary search steps will be needed in the last-mile search. This table can further be used as input to the reference implementation to directly generate code for each table entry.

\section{Adaptation for P4}
Until now, the discussed RMI implementation for BMv2 in chapter \ref{ch:rmionbmv2} was extremely unflexible and solely focussed on a single dataset where every configuration or change was done by hand in a quite uncomfortable way. This chapter is about going a step further, where an adaptation of the reference RMI implementation in Rust should potentially be able to automatically generate P4 source code files depending on which input dataset was targeted and what models were selected. This should not only allow an RMI implementation in P4 for the books\_200M dataset with 32-bit keys, but instead many different configurations for all provided SOSD datasets hopefully become executable on the BMv2 switch.\\

An important thing to mention is that the entire mathematical or learned part of the reference RMI implementation will stay completely untouched and only the code generation part of the reference implementation will be adjusted.

\section{Code generation}
Generally, when looking at the fully implemented final result from chapter \ref{ch:rmionbmv2} a lot of static code is to be found in the P4 file which stays the same for any dataset or model combination. Coyping all these header definitions or the normalization function and other helper functions into a P4 source file is straightforward. A first step that remains is to treat code generation for different mathematical helper functions depending on which models are used. An important property here is that a function should only be printed into the result file if it is actually needed. This is covered in more detail in Sections \ref{sect:rmiforp4:linear_cubic} and \ref{sect:rmiforp4:radix}. A next thing to treat is code generation of the actual lookup function which has to adapt with respect to different model combinations. This is the centerpiece and most complex part of the code generation and discussed more in detail in Section \ref{sect:rmiforp4:lookup}. Finally the code generation that makes sure that model parameters can be sent over to the switch via P4Runtime in Python or when small enough statically printed into the result source file remains. This part of the implementation proved to be more complex than initially thought since the generated Pyhton source file for P4Runtime has to seamlessly work with the table declaration printed into the P4 source file. This is looked at in more detail in Section \ref{sect:rmiforp4:parameters}.

\subsection{For linear and cubic models}
\label{sect:rmiforp4:linear_cubic}
The inner workings of the lookup functions for both of these models were already covered in Section \ref{sect:rmionbmv2:implmentation} and \ref{sect:rmionbmv2:fma} together with the concrete implementation shown in the appendix in Section \ref{sect:appendix:fma}. The only thing left for the code generation in the proposed implementation apart from printing said code into the P4 source code file is to make sure that either static model parameters are correctly printed into the source file or that larger amounts of model parameters are correctly loaded during the switches runtime. As already stated this is looked at from a more general point of view in Section \ref{sect:rmiforp4:parameters}.

\subsection{For radix models}
\label{sect:rmiforp4:radix}
The model function for radix models is the only one that does not rely on floating point arithmetic and is therefore more amenable to work in P4. The reference RMI implementation contains two radix models. The first of them uses a certain prefix length to bit shift on the input to generate a radix value which is directly the resulting output of the model, whereas the second model calculates a radix value based on the input the same way but instead uses it then to index a radix table which then serves as the resulting output of the model. For these models, the adaptation into P4 and code generation is even simpler since all necessary primitive operations used are also available in P4. The second radix model involving a radix table though needs a bit more consideration which involves loading the radix table using the mechanisms described in \ref{sect:rmiforp4:parameters} and adapting the lookup function generation accordingly.

\pagebreak

\subsection{For the lookup function}
\label{sect:rmiforp4:lookup}
The lookup function is probably the most important but with that also the trickiest part of the code generation implementation. In this part all sorts of combinations of models as well as other properties of the learned RMI must be considered. Generally due to the nature of RMI, in the sense that each model layer's output provides the input index for the next layer, the generation code works in the same way by looping over each generated layer. For each layer based on model properties, it is decided if floating point or integer input and output is needed. Based on this information, conversions between layers are added if necessary. Further, there is a difference between code generation for the first layer and all following layers. Theoretically, several following model layers are possible but the reference implementation, and this work, as previously stated, focus on only having a single following layer from now on designated as the second layer. When generating code for the first layer, the model parameters originally printed into a header file are now converted to the customly defined floating point format in P4 and statically written into the resulting source code file as input for the model function. One exception to this being the model involving a radix table, where additionally to the call of the model function a table lookup into the model parameters table loaded with the mechanism described in the next section happens. Even though this is already implied, in both cases the code generation inserts a call to the respective model function at the appropriate location. Next, when generating code for the second layer, a call to the function that looks into the model parameters table to retrieve the corresponding model parameters gets written to the source code file with the resulting index from the previous layer as input. Finally, a call to the second layer model function is appended with the just retrieved model parameters as input arguments. At the end a function to calculate the final result by clamping it to a value between 0 and the dataset size is appended. With that the generation of the lookup function is complete.

\subsection{For loading model parameters}
\label{sect:rmiforp4:parameters}
The RMI reference implementation dumps larger chunks of model parameters into a binary file. The idea of saving a binary file containing all layer parameters for later use is kept by our implementation. In order to load these parameters properly for each layer, multiple source code files are concerned. The first one being the table declaration itself in the P4 source code file. A table declaration and a corresponding table lookup function are added to the P4 source file whenever a layer needs to load model parameters. The generation of this function takes into account how many parameters need to be loaded and how large the table is going to be based on the learned RMI layer properties. Finally, the generated table lookup function can be used at the appropriate position in the lookup function in order to retrieve layer parameters for a specific model index. The second file has to be generated in Python and uses P4Runtime \cite{p4runtime-spec}. It has to load the previously stored layer parameters from said binary file and then fill the previously declared table. It does so by creating a table entry for each model parameter loaded from the binary file and sending these in batches to the switch. The generation of this source code file is in the same way dynamic as the table declaration, in the sense that the generated source code will adapt depending on how many model parameters each generated RMI layer needs.

\section{How to run the generated code}
All together this creates a process where, based on a learned RMI configuration, the proposed implementation in this chapter can output a P4 and a Python source code file. The idea is to first setup the virtual network via Mininet that contains at least one switch which is executing the freshly generated P4 source code file. When everything is operating correctly the generated Python source code file can be run in order to load model parameters from the saved binary file and send them to the switch using P4Runtime. Finally after this process is complete the switch is able to successfully respond to incoming learned RMI packets formatted as described in the previous chapter in Section \ref{sect:rmionbmv2:network}.

\section{Supporting other models}
Currently the proposed adaptation does only support the three model types described in Sections \ref{sect:rmiforp4:linear_cubic} and \ref{sect:rmiforp4:radix}. The reference implementation, on the other hand, supports additional model types like normal, logarithmic or histogram. There are multiple reasons for why these models are currently not supported.\\

A first reason, and probably the most important one, being that these models are hard to implement in P4 because they all rely on some additional mathematical functions which in turn rely on some currently not implemented floating point arithmetic operations. An example being the exponential functions which are used for the normal or logarithmic model types. In my opinion, there is no reason why a similar approach than the one taken for the FMA instruction would not work. In other words, a software implementation of said mathematical floating point arithmetic functions would probably be doable in P4. Currently though, since time is short and since this project does not necessarily have the goal of being a full copy of the reference implementation, these model types are left aside for future work. Lastly, from a more practically oriented point of view, when looking at all pareto optimal configurations generated by the reference RMI implementation, most of them rely on some combination of linear and cubic model types.
