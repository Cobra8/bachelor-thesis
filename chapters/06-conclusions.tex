% !TEX root = ../main.tex

\chapter{Conclusion}
\label{ch:conclusion}
In this work we started by evaluating different learned index structures by their potential programmability in P4. We compared performances of different (learned) index structures using the SOSD benchmark \cite{sosd-neurips} and looked at what is possible as well as what the limits of network programmability are. We came to the conclusion that the RMI learned index structure initially proposed in \cite{rmi} is a good fit for further pursuing our idea of implementing a learned index structure in P4 in order to run lookup calculations on the fly over the network.\\

In a next step we implemented a proof of concept by hand in P4 testing it's operability on virtually simulated network hardware and found that an RMI implementation with perfect accuracy in P4 is indeed possible. At the same time though, we learned what limitations apply on real world switch hardware that do not exist in simulated hardware which proved to be crucial for our implementation. This leads to probably the most pertinent conclusion of this work, namely that in order for learned indices to become viable solutions in real world applications in the future, switches need to be able to support floating point arithmetic ideally on their ALUs or even FPUs or in some other computationally cheap form. In the same sense another limitation which we are facing, especially strengthened due to the fact that we naïvly implemented floating point arithmetic in software, is that real world switches are limited by the amount of ALU stages, meaning that chained operation complexity for a single packet is pretty quickly exploited. After having implemented a proof of concept that worked for a specific dataset and a specific RMI configuration, we further opted for a more generalizable solution, by adapting the RMI reference implementation \cite{cdfshop} in a way that P4 source code files can be generated for any input dataset and mostly any RMI configuration.\\

Finally in a last step, we tried to circumvent the fact that our implementation is theoretical and tried to estimate, how much an RMI implementation over the network could save in a closed system. We come to the conclusion that when assuming ideal conditions, where an RMI implementation in P4 exists under which a switch can operate at state-of-the-art packet processing rates in an application where a lookup has to be sent over the network anyways, our idea could potentially leverage each last mile search worker by a constant magnitude of around 50-100ns per lookup depending on the RMI layer configuration.

\pagebreak

\section{Future Work}
The continuation of this work would initially consist of finding solutions for the different limitations and assumptions that were taken in the scope of this work in order to arrive at a real world learned index structure implementation over the network. One solution for floating point arithmetic could come for instance, from an immensely interesting paper \cite{netfc} about enabling accurate floating point arithmetic on P4 capable switches in a smarter way than we did for this work. Unfortunately, the results are currently limited to 16-bit floating point arithmetic and everything beyond is left as future work. Further there is another paper \cite{maltry2021critical} which proposes an alternative implementation for finding RMI configurations that yield similar lookup times and often better build times. The result of this alternative implementation could be used as input to the P4 code generation part of our adaptation of the reference implementation. This does not immediately lead to a solution of one of the problems shown in this work, but could further leverage the performance of RMI. Finally when looking at all limitations and assumptions that we discovered during this work, most of them could either be solved or at least greatly weakened when having hardware supported floating point arithmetic. In that sense a very systems related but extremly interesting start for our future work could be to try to extend the capabilities of network switches or potentially NICs to support floating point arithmetic. The topic is currently already quite highly frequented and there are already papers \cite{inline-fpa} which pursue this exact idea of applying hardware enhancements in order to allow efficient floating point arithmetic.\\

Generally evaluating other or in the future also new learned index structures regarding their network programmability will continuously be a good starting point for our future work. In that sense pursuing the idea of having a learned index structure which does not rely on floating point arithmetic could be worth exploring.
